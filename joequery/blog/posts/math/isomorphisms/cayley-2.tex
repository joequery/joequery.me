\documentclass{article}
\begin{document}
First, we need to verify that $T_g \colon G \to G$ is well defined. We know the domain of $T_g$ is all of $G$ by how we defined $T_g$. The range is at least a subset of $G$ since $gx \in G \mbox{ for each }x \in G$. So $T_g$ is well defined.

We now verify $T_g$ is 1-1. Let $x,y \in G$. Suppose $T_g(x)=T_g(y)$. Then
\begin{eqnarray*}
 gx &=& gy \\
 x &=& y
\end{eqnarray*}
Since $T_g(x)=T_g(y)$ implies $x=y$, $T_g$ is 1-1.

We now verify $T_g$ is onto. Let $y \in G$. Note $g^{-1} \in G$ since G is a group, and $g^{-1}y \in G$ since $G$ is closed. Observe:
\begin{eqnarray*}
T_g(g^{-1}y)&=&g(g^{-1}y) \\
&=& (gg^{-1})y \\
&=& ey \\
&=& y
\end{eqnarray*}
So for an arbitrary $y \in G$, $\exists x \in G$ such that $T_g(x)=y$, so $T_g$ is onto. Since we have shown $T_g$ is a well defined mapping from $G$ to $G$ that is 1-1 and onto, we have shown that $T_g$ is a permutation on $G$.
\end{document}
